\documentclass[journal,review,submit,pdftex,moreauthors]{Definitions/mdpi} 
\usepackage[sorting=none]{biblatex}
\usepackage{graphicx}
\usepackage{amssymb}
\usepackage{booktabs}
\usepackage{multirow}
\usepackage[normalem]{ulem}
\useunder{\uline}{\ul}{}
\addbibresource{bibliography.bib}

% MDPI internal commands - do not modify
\firstpage{1} 
\makeatletter 
\setcounter{page}{\@firstpage} 
\makeatother
\pubvolume{1}
\issuenum{1}
\articlenumber{0}
\pubyear{2023}
\copyrightyear{2023}
%\externaleditor{Academic Editor: Firstname Lastname}
\datereceived{ } 
\daterevised{ } % Comment out if no revised date
\dateaccepted{ } 
\datepublished{ } 
%\datecorrected{} % For corrected papers: "Corrected: XXX" date in the original paper.
%\dateretracted{} % For corrected papers: "Retracted: XXX" date in the original paper.
\hreflink{https://doi.org/} % If needed use \linebreak

\Title{Fourier ptychographic microscopy 10 years on: A review}

% MDPI internal command: Title for citation in the left column
\TitleCitation{Title}

% Author Orchid ID: enter ID or remove command
\newcommand{\orcidauthorA}{0000-0000-0000-000X} % Add \orcidA{} behind the author's name
%\newcommand{\orcidauthorB}{0000-0000-0000-000X} % Add \orcidB{} behind the author's name

% Authors, for the paper (add full first names)
\Author{Fannuo Xu $^{1,2,\dag}$, Zipei Wu $^{1,3,\dag}$, Chao Tan$^{1,4}$, Yizheng Liao$^{1,2}$, Keru Chen$^{1,5}$ and An Pan$^{1,2,*}$}

%\longauthorlist{yes}

% MDPI internal command: Authors, for metadata in PDF
\AuthorNames{Fannuo Xu, Zipei Wu, Tan Chao, Yizheng Liao, Keru Chen and An Pan }

% MDPI internal command: Authors, for citation in the left column
\AuthorCitation{Lastname, F.; Lastname, F.; Lastname, F.}
% If this is a Chicago style journal: Lastname, Firstname, Firstname Lastname, and Firstname Lastname.

% Affiliations / Addresses (Add [1] after \address if there is only one affiliation.)
\address{%
$^{1}$ \quad State Key Laboratory of Transient Optics and Photonics, Xi’an Institute of Optics and Precision Mechanics,Chinese Academy of Sciences, Xi’an 710119, China; xufannuo22@mails.ucas.ac.cn(F.X.),liaoyizheng22@mails.ucas.ac.cn(Y.L.)\\
$^{2}$ \quad University of Chinese Academy of Sciences, Beijing 100049, China; \\
$^{3}$ \quad School of Physics and Optoelectronic Engineering, Shenzhen University, Shenzhen 518060, China; shadowbinyomin@gmail.com \\
$^{4}$ \quad School of Electronics and Information Engineering, Sichuan University, Chengdu 610065, China; tanchao@stu.scu.edu.cn \\
$^{5}$ \quad School of Automation Science and Engineering, Xi'an Jiaotong University, Xi'an 710049, China; chenkeru@stu.xjtu.edu.cn \\
}

% Contact information of the corresponding author
\corres{Correspondence: panan@opt.cn}

% Current address and/or shared authorship
\firstnote{These authors contributed equally to this work.} 

% The commands \thirdnote{} till \eighthnote{} are available for further notes

%\simplesumm{} % Simple summary

%\conference{} % An extended version of a conference paper

% Abstract (Do not insert blank lines, i.e. \\) 
\abstract{In 2013, Fourier ptychographic microscopy (FPM) emerged as a groundbreaking imaging technique, garnering widespread attention due to its remarkable features: high resolution, wide field-of-view (FOV), and quantitative phase recovery. Over the past decade, FPM has evolved into a pivotal tool in microscopy, finding applications in diverse fields such as biomedicine, scientific research, and inspection and metrology. This evolution is rooted in its ability to address the long-standing challenge of balancing resolution and FOV in imaging systems. In this comprehensive review, we delve into the fundamental principles of FPM, drawing comparisons with related imaging modalities. Furthermore, we explore various experimental implementations and highlight key milestones in four crucial technological aspects: speed, three-dimensionality, color imaging, and the integration of deep learning. As a high-throughput optical imaging technique, FPM offers a multitude of applications, ranging from digital pathology and drug screening to label-free imaging.
Despite its already impressive accomplishments, we emphasize that FPM is still in its nascent stages, leaving ample room for further advancements. We conclude by discussing the pertinent challenges that lie ahead and the promising future applications of FPM.}

% Keywords
\keyword{Fourier ptychographic microscopy; computational imaging; biomedical imaging;} 
%%%%%%%%%%%%%%%%%%%%%%%%%%%%%%%%%%%%%%%%%%
\begin{document}

%%%%%%%%%%%%%%%%%%%%%%%%%%%%%%%%%%%%%%%%%%
\setcounter{section}{0} %% Remove this when starting to work on the template.

\section{Introduction}

Microscopes serve as a critical tool for exploring the microcosm and have played an indispensable role in the advancement of human technology and civilization. Optical microscopes, in particular, achieve imaging by capturing and processing the interactions between light and the sample. Due to their non-invasive and non-contact advantages, they are widely used in fields such as biology and medicine. With the rapid development of digital pathology, flow cytometry, and high-throughput screening technologies, the demand for high-resolution and large FOV imaging is increasing, prompting continuous advancements in optical microscopy techniques.

Facing this challenge, a traditional solution is to increase the aperture of the objective lens to expand the FOV. However, this significantly increases manufacturing costs and complicates the control of moving and focusing the objective lens. Furthermore, this method induces geometric distortions that necessitate additional lenses for correction, making it suitable only for fixed scenarios such as steppers.

Another simple yet effective method involves using high-magnification objectives to image the sample at different positions and then stitch the scans together. This approach ensures high resolution while expanding the FOV. However, the quality of the stitched image is often poor, with artifacts like ghosting and blurring. Additionally, the scanning process is time-consuming, and the high-precision motorized stage and high-magnification objectives substantially increase manufacturing costs.

Synthetic aperture radar technology\cite{turpin1995theory,holloway2017savi}is an effective way to improve resolution in the microwave band. However, due to the much higher frequency of visible light compared to microwaves, phase measurement becomes exceptionally challenging, limiting its application in the biomedical field.

Lensless imaging technology\cite{zhang2019lens}, developed in recent years, has advantages like small size, large FOV, and low cost. It performs particularly well in scenarios where lenses are difficult to manufacture, such as X-ray imaging\cite{thibault2008high}. However, its potential for resolution improvement in the optical spectrum is relatively limited, and it struggles with phase recovery in low-frequency regions\cite{zhang2019lens}. Therefore, the application value is limited.

In summary, all four methods for achieving high resolution and a large FOV have certain issues and limitations in practical applications. For microscopes, most biological samples under observation are colorless and transparent. Traditional microscopes can only capture low-contrast images as they record only the light intensity distribution within the FOV. However, the structural information of the sample can be represented by the phase information of the transmitted light\cite{thorn2016quick}, making phase measurement a development goal for microscopy techniques.

An imaging technique that can restore phase and obtain high-resolution images is called digital holography\cite{javidi2021roadmap}, which utilizes the principle of interference. This technique involves the diffraction calculation of the interference pattern between the object wave and the reference wave, thereby reconstructing the hologram. To enhance imaging resolution and signal-to-noise ratio, the concept of synthetic aperture was adopted, leading to the invention of synthetic aperture digital holography\cite{le2001synthetic,massig2002digital}. However, it is still constrained by the physical law between the FOV and resolution, resulting in a limited spatial bandwidth product. Moreover, due to its interference-based principle, the imaging apparatus is quite complex. In addition to DH, techniques such as self-interference digital holography\cite{rosen2007digital} and optical coherence tomography\cite{huang1991optical,podoleanu2012optical} also directly utilize light interference to measure the morphology of samples, and are referred to as imaging interferometric microscopy.

Another technique related to phase retrieval is called ptychography\cite{hoppe1969beugung}, which was first proposed by Hegerl et al. in 1969 as a form of coherent diffraction imaging to solve phase measurement issues in electron microscopy. This method has been extended to imaging in multiple bands over its subsequent development. In 2004, Rodenburg et al. introduced the ptychographic iterative engine(PIE)\cite{rodenburg2004phase,faulkner2004movable} algorithm, which achieved large field-of-view, high-resolution imaging by using multiple overlapping diffraction patterns in the spatial domain and recovered the phase using iterative algorithms. Maiden and Rodenburg proposed ePIE\cite{maiden2009improved} in 2009, which can solve for both the sample distribution function and the probe function. To accelerate the reconstruction speed, Clark introduced fly-PIE\cite{clark2014continuous}. In 2012, Maiden introduced the 3PIE\cite{maiden2012ptychographic} algorithm, making ptychography applicable for thick three-dimensional(3D) samples. In 2017, by improving the iterative process of ePIE using convex combination and introducing regularization weighting, the regularized PIE\cite{maiden2017further} was further proposed by Maiden and colleagues. Furthermore, by introducing momentum, they proposed the momentum-accelerated PIE\cite{maiden2017further}. Additionally, for X-ray\cite{pfeiffer2018x} wavelengths that are difficult to fabricate with precision imaging devices, the imaging mode of ptychography offers significant advantages. For instance, ptychographic computed tomography\cite{dierolf2010ptychographic,diaz2012quantitative} was used to obtain high-resolution and quantitative 3D insights.

Inspired by significant advancements in optical imaging mentioned above, especially IIM's innovative use of synthetic aperture in optical bands and ptychography's unique iterative algorithm that alternates constraints in both the frequency and spatial domains, Zheng et al. introduced FPM\cite{zheng2013wide} in 2013. This technique, deeply rooted in the principles of the aforementioned methods, blends the spatial resolution enhancement of synthetic aperture with the computational capabilities and phase retrieval of ptychography. FPM not only offers high-resolution and expansive FOV imaging but also excels in quantitative phase imaging, gleaning crucial data from diffraction patterns. Furthermore, FPM integrates the concepts of phase recovery and synthetic aperture, marrying LED array illumination with PIE phase recovery algorithms. It sidesteps the traditional step-scanning operation in ptychography by transitioning the stitching operation to the frequency domain through multi-angle illumination\cite{horstmeyer2014phase}. In essence, FPM captures a series of low-resolution images under varied illumination angles, leveraging intensity information in the spatial domain and the cutoff frequency in the frequency domain as constraints. The technique then stitches these in the frequency domain while concurrently recovering the phase using iterative algorithms.

Over the past decade, FPM technology has rapidly developed. Figure~\ref{data} illustrates the evolution of related publications since its inception in 2013, with notable milestones in its development highlighted within the figure. It is not only a means of microscopic imaging but has also become a computational imaging framework for dealing with optical system limitations. It has changed the traditional "what you see is what you get" imaging mode by combining the encoding of optical images with digital decoding, achieving a significant improvement in imaging quality, breaking the physical limitations of optical systems, and greatly enhancing the information acquisition capability of imaging systems.

This review aims to provide a comprehensive introduction to the basic principles of FPM technology and compare it with other microscopic imaging technologies, such as structured illumination microscopy (SIM), synthetic aperture, and differential phase contrast (DPC). We will also systematically summarize various applications and technological advancements based on FPM, including high-speed imaging, high-throughput screening, 3D imaging, color imaging, and the integration of machine learning. Finally, we explore the application prospects of FPM in digital pathology, drug screening, and other fields, and look forward to its future development.

\begin{figure}[H]
\includegraphics[width=13.7cm]{pictures/data.png}
\caption{The advancement of the FPM has seen a remarkable growth in the number of related publications since its inception in 2013. Notable milestones in its development have been highlighted.\label{data}}
\end{figure}  

%%%%%%%%%%%%%%%%%%%%%%%%%%%%%%%%%%%%%%%%%%
\section{Principle and comparison}
%--------------------------------------------------------------------

\begin{figure}[H]
    \centering
    \includegraphics[width=13.8cm]{pictures/comparison_v2.3.png}
    \caption{\textbf{(a)}: FPM implementations using LED arrays.
    \textbf{(b)}: Conventional ptychography: lensless configuration with coded illumination.
     \textbf{(c)}: The synthetic aperture process is performed using camera scanning.
     \textbf{(d)}:  Schematic  diagram  of  structured  illumination  microscopy.
     \textbf{(e)}: Coded ptychography: lensless configuration with coded detection.
     \textbf{(g)}: Transport of intensity equation setup.
     \textbf{(f)}: By collecting four semicircular illumination intensity images and subtracting them in pairs to obtain phase contrast images along the x and y axes, DPC can achieve quantitative phase images through one step of deconvolution.
    }
    \label{fig:enter-label}
\end{figure}


\subsection{Principle}
The traditional FPM technique collects low-resolution images at different illumination angles and calculates the high-resolution image through iterative algorithms. According to the theory of Fourier analysis, multi-angle illumination corresponds to the translation of the sample spectrum. Considering the low-pass characteristics of the coherent transfer function (CTF) of the imaging system, the translation of the sample spectrum means that the originally truncated high-frequency information is shifted into the passband and is thus recorded.

In the physical model, let the wave vector of the plane wave incident at an oblique angle into the sample be $(k_{x,m,n},k_{y,m,n})$, where $m,n$ correspond to the $m^{th}$ row and $n^{th}$ column of the LED array. The object function of the sample is $o(x,y)$, representing the complex amplitude transmittance. When the sample can be considered as a "thin sample" the transmitted light field is $o(x,y)e^{(jxk_x,jyk_y)}$. After low-pass filtering by the CTF of the optical system, the intensity distribution $I_{m,n}(x,y)$ recorded by the camera can be written as:
\begin{equation}
\begin{split}
I_{m,n}(x,y) &= |\mathscr{F}^{-1}\{\mathscr{F}\{o(x,y)e^{(jxk_{x,m,n},jyk_{y,m,n})}\}P(k_x,k_y)\}|^2 \\
&= |\mathscr{F}^{-1}\{O(k_x-k_{x,m,n},k_y-k_{y,m,n})\}P(k_x,k_y)|^2
\end{split}
\end{equation}
Here, $O(k_x, k_y)$ is the Fourier transform of the object function, $\mathscr{F}$ and $\mathscr{F}^{-1}$ are the Fourier transform and inverse Fourier transform, respectively, and $P(k_x,k_y)$ is the CTF of the optical system. For an ideal imaging system, the CTF can be expressed as:
\begin{equation}
P(k_x,k_y) = 
\begin{cases} 
1 & \text{if } k_x^2+k_y^2\leq k_c^2 \\
0 & \text{if } k_x^2+k_y^2 > k_c^2
\end{cases}
\end{equation}
Here, $k_c$ is the cutoff frequency of the optical system. When the numerical aperture of the objective lens is $NA_{obj}$ and the wavelength of the illuminating light is $\lambda$, $k_c$ can be written as $k_c=2\pi \frac{NA_{obj}}{\lambda}$.

Therefore, the core objective of FPM technology is to find an optimal object function that best explains the intensity distribution observed by the camera under different angle illuminations. This can be achieved through the following optimization problem:
\begin{equation}
\mathop{\arg\min}_{O(k_x,k_y)} \ \ \sum_{m,n}\sum_{x,y}|\sqrt{I_{m,n}(x,y)}-|\mathscr{F}^{-1}\{O(k_x-k_{x,m,n},k_y-k_{y,m,n})\}P(k_x,k_y)||^2
\end{equation}
The traditional FPM uses the Gerchberg-Saxton algorithm \cite{gerchberg1971phase}, also known as the alternating projection method, to solve the above non-convex optimization problem. The algorithm includes the following key steps:

\begin{enumerate}
    \item \textbf{Initial Guess}: Guess the frequency domain of the high-resolution object function after Fourier transform as $O_0(k_x,k_y)$. The initial guess of the amplitude of the object function spectrum is generally obtained by upsampling the low resolution collected during vertical illumination, and its phase is generally initialized with all zeros.
    
    \item \textbf{Frequency Domain Support Constraint}: For the LED illumination at coordinates $(m,n)$, the corresponding illumination wave vector is $(k_{x,m,n},k_{y,m,n})$. Apply low-pass filtering to the predicted object function spectrum in the frequency domain and perform its inverse Fourier transform to obtain the low-resolution spatial image.
    \begin{equation}
    o^e_{m,n}(x,y)=\mathscr{F}^{-1}\{O_0(k_x-k_{x,m,n},k_y-k_{y,m,n})P(k_x,k_y)\}
    \end{equation}
    Here, the superscript $e$ indicates that the complex amplitude of this object function is to be updated.
    
    \item \textbf{Spatial Amplitude Constraint}: Keep the phase of the object function unchanged and replace the amplitude of the object function with the amplitude of the measured low-resolution image, i.e.,
    \begin{equation}
    o^u_{m,n}(x,y)=\sqrt{I^c_{m,n}(x,y)}\frac{o^e_{m,n}(x,y)}{|o^e_{m,n}(x,y)|}
    \end{equation}
    Here, $I^c_{m,n}(x,y)$ is the low-resolution image measured when the LED at coordinates $(m,n)$ is illuminated, and the superscript $u$ indicates that the complex amplitude of this object function has been updated.
    
    \item \textbf{Update Object Function Spectrum}: After the updated object function is constrained in the spatial domain, it is Fourier transformed into the frequency domain. In the original guessed spectrum, replace the spectrum corresponding to this angle of illumination in the passband, and keep the spectrum at other positions unchanged, i.e.,
    \begin{equation}
    O(k_x,k_y)=\mathscr{F}\{o^u_{m,n}(x,y)\}P(k_x,k_y)+\mathscr{F}\{o^e_{m,n}(x,y)\}[1-P(k_x,k_y)]
    \end{equation}
    
    \item \textbf{Sub-aperture Update}: Replace the obtained $O(k_x,k_y)$ with the initial guess and repeatedly update the sub-apertures corresponding to all illumination angles.
    
    \item \textbf{Iterative Repetition}: Repeat steps 2-5 until the high-resolution spectrum obtained converges. After the inverse Fourier transform, the reconstructed high-resolution object function, including its amplitude distribution and phase distribution, can be obtained.
\end{enumerate}

The above algorithm is the original FPM reconstruction algorithm proposed by Zheng et al. in 2013. In the subsequent 10 years, researchers have proposed various more robust reconstruction algorithms to improve the reconstruction quality of FPM. In 2014, Ou et al. proposed the EPRY-FPM\cite{ou2014embedded} algorithm, which can simultaneously reconstruct the object function and the pupil function. Zuo et al. proposed an adaptive variable step-length iterative method\cite{zuo2016adaptive} in 2016, which has better robustness. Subsequently, optimization methods based on first-order and second-order gradients were successively proposed, including the Gauss-Newton method\cite{tian2014multiplexed}, global quasi-Newton method\cite{yeh2015experimental}, and Wirtinger flow method\cite{bian2015fourier}.In 2022, Wang et al. proposed a FPM reconstruction algorithm based on the alternating direction method of multipliers (ADMM), called ADMM-FPM\cite{wang2022fourier}. This algorithm decomposes the FPM reconstruction problem into multiple subproblems, which solve for the object function, the probe function, and the Lagrange multipliers separately, and updates them using the alternating direction method. This algorithm has a high degree of parallelism, which enables efficient execution on GPU or distributed CPU. This algorithm also introduces an adaptive step size strategy, which adjusts the step size dynamically according to the error in the reconstruction process, improving the convergence speed and stability. This algorithm is compared with existing FPM reconstruction algorithms in simulation and experiment, and the results show that ADMM-FPM algorithm has better robustness and reconstruction quality under noise interference. This algorithm provides a novel and effective reconstruction method for FPM.
In recent years, algorithms based on sparse representation\cite{zhang2017fourier}, fast iterative shrinkage/threshold algorithm\cite{sun2019regularized}, and low-rank matrix recovery\cite{jagatap2019sample} have been proposed, further optimizing the imaging quality of FPM. With the development and rise of deep learning, its application in reconstruction algorithms has become increasingly widespread and profound. Both data-driven models and physical models have achieved good results in image reconstruction. We will delve into this topic in the subsequent sections.

FPM technology reconstructs images through sub-aperture overlap. The scattered light, resulting from the interaction of light with the sample outside the numerical aperture of the objective lens, is collected by the objective lens. This process produces dark-field images that contain high-frequency information exceeding the cutoff frequency of the optical system. The lateral resolution is rewritten from the traditional $d=\frac{\lambda}{2NA}$ to $d=\frac{\lambda}{NA_{syn}}=\frac{\lambda}{NA_{illu}+NA_{obj}}$, where $NA_{illu}$ and $NA_{obj}$ respectively represent the illumination and objective NA. FPM has achieved a breakthrough in having both a large FOV and high resolution. 


%-------------------------------------------------------------------
\subsection{Ptychography}
The earliest pytchography was used for crystallography measurements\cite{hoppe1982trace}. Later, it was gradually used in the field of biological imaging due to its lower requirements for optical system hardware and the ability to achieve higher-resolution images. The conventional ptychography structure is shown in the figure (to be added later). The range of the illumination detection beam is limited through a spatially restricted aperture, and the image sensor is placed in the far field for data collection. During the data acquisition process, the probe beam is moved to different spatial positions and the diffraction pattern at each position is recorded in the far field. Since the images collected by the probe beam at different positions have a certain overlap, under the premise of ensuring a certain overlap rate, the consistency constraints of the overlapping areas can be used to calculate the relative phases of adjacent positions, and finally reconstruct the object and Complex amplitude distribution of the probe beam. In the process of its reconstruction, the first set of constraints is called the support domain constraints, which is achieved by setting the signal outside the probe beam area to zero and leaving the signal in the area unchanged; the second set of constraints is called the Fourier amplitude The constraint is achieved by replacing the modulus of the estimated value with the measured value while keeping the phase constant. Iterate continuously under two sets of constraints to achieve the restoration of the complex amplitude of the image\cite{wang2023optical}. On the basis of pytchography, the FPM technology introduced in this article and many variants subsequently emerged. For example, the pytchography X-ray computed tomography technology combined with X-ray computed tomography can realize nanometer analysis of 3D samples, the reconstruction of internal structure and phase information with sub-angstrom resolution, and the ptychographic electron microscopy technology combined with scanning electron microscopy can achieve sub-angstrom resolution phase imaging of electron projection samples\cite{dierolf2010ptychographic,humphry2012ptychographic}.

%--------------------------------------------------------------------
\subsection{Lensless imaging technology }
Lensless imaging technology has emerged as a prominent imaging method in recent years, with its main advantages including small size, large FOV, and low cost. This technology does not use lenses, but instead employs sensors to capture the diffraction patterns generated by the object. It finds extensive applications in scenarios where lenses are difficult to manufacture, such as X-ray imaging. However, in the visible light spectrum, traditional lens-based technologies are quite mature, and microscopy techniques based on traditional microscope structures often yield better imaging results. Moreover, due to limitations in semiconductor technology and light collection efficiency, the sensor pixel size cannot be infinitely reduced, limiting the potential for resolution improvement in lensless imaging\cite{zhang2019lens}. Additionally, this technology has special requirements for sample preparation, making its application range relatively narrow.

Nevertheless, the advantages of lensless imaging have been incorporated into FPM technology. For example, in 2015, researchers like Luo successfully implemented FPM in a lensless environment by moving a fiber-optic probe to provide multi-angle illumination\cite{luo2015synthetic}. Zhang and others achieved higher-resolution lensless FPM technology in 2023\cite{zhang2023lensless}. They adjusted the position of the Fresnel zone aperture in front of the sensor and filled the extracted real spectrum in the frequency domain. They successfully achieved lensless super-resolution imaging reconstruction based on FPM, achieving a threefold improvement in resolution without considering diffraction.

%--------------------------------------------------------------------
\subsection{Synthetic aperture}
Synthetic aperture imaging is a significant technology that was developed to address resolution limitation imposed by imaging systems through multi-angle object scanning \cite{turpin1995theory, holloway2017savi}. And this concept originated from synthetic aperture radar \cite{ curlander1991synthetic} and is extensively applied into phase measure. In optical wavelengths, there are many researches which combine the concept with digital holography and microscopy imaging\cite { schwarz2003imaging, alexandrov2006synthetic, mico2006synthetic, kreis2007resolution, di2008high,hillman2009high, tippie2011high, mico2019resolution}. However, these traditional synthetic imaging record the variable-angle complex amplitude of the light wave, subsequently arranging this information in Fourier space and require a highly coherent light source and a rigorously stable experimental environment. Besides, the speckle noise is a significant adverse effect reducing the quality of images. In contrast, FPM initially records only intensity information, omitting phase data, and then iteratively recovers the phase from redundant information without interferometric methods. In summary, FPM and DHM share objectives but differ in data acquisition techniques, with FPM's iterative approach offering advantages in simplifying experiment setup and operations.
Non-interferometric and non-iterative phase imaging


%--------------------------------------------------------------------
\subsection{Transport of intensity equation and Kramers-Kronig relations}

Transport of intensity equation (TIE) is a mathematical approach used to extract phase information from light intensity measurements which was introduced by Teague in 1983 \cite{ teague1982irradiance,teague1983deterministic}. In the TIE method, it is necessary to collect the light intensity of at least two defocusing surfaces for finite difference estimation to obtain the axial differential of the light intensity on the left side of the equation. Besides, the light intensity distribution of the plane to be measured can be obtained through direct measurement, and then the phase distribution information of the plane to be measured can be obtained by numerical solution without any iterative process \cite{paganin1998noninterferometric}. TIE limits the imaging range and adopts approximation pieces, achieving the quantitative phase imaging in the form of analytics. Compared with other methods like FPM, the method has the advantages of simple calculation, no phase unwrapping, no complex optical system, etc., and has shown broad application prospects in the field of optical microscopy imaging and metrology\cite{ zuo2020transport}. But, it requires specific conditions like near-axial and weak defocusing to establish a linear relationship between intensity and phase and its phase measurement accuracy is not as high as interferometric methods. Moreover, it's applicable mainly in situations like near-field Fresnel diffraction and not suitable for situations like Fraunhofer diffraction at large angles. In the past years, Li et al. combined TIE with FPM to realize 3D label-free microscopy without interferometric synthetic aperture\cite{li2022transport}.

Kramers-Kronig (KK) relations is another non-interferometric and non-iterative phase imaging method which describes the interconnection between the real and imaginary parts of the frequency domain response of a linear system. KK relations is a specific case of the Hilbert transform and is used for square-integrable functions with causal relationships \cite{chang2023robust}. Park et al. introduced a high spatial bandwidth product (SBP) off-axis holographic imaging technique based on the KK relationship. This method doesn't impose assumptions on the sample and accurately retrieves complex amplitude information. It offers significant improvements in SBP compared to traditional methods \cite{baek2021intensity}. Huang et al. combined KK relationship with holographic multiplexing technology, breaking through the limit of bandwidth utilization of traditional technology \cite{huang2022high}. Cheng et al. proposed synthetic aperture imaging based on Kramers–Kronig relations (KKSAI) technology, which collects images through a modified microscope system with pupil modulation capability \cite{shen2021non}. The phase and amplitude profile of the sample at pupil limited resolution can be extracted from as few as two intensity images by using KK relations. Li et al. presented a quantitative phase method named AIKK which is accomplished by citing KK relations to combine measurements captured by an LED array microscope with programmable annular-illumination source patterns \cite{li2021fast}. Compared with pupil modulation FPM \cite{ou2016aperture,lu2016quantitative}, these methods based on KK relations require much fewer measurements or any sample priors and iteration free. But KK relations is very demanding for numerical aperture (NA) matching, which makes the experimental operation need to be accurate. And the KK relationship can achieve twice the resolution but there is no room for further improvement.

TIE and KK relation, two direct phase recovery methods, do not require iterative operation and interference, so the speed of solving is their common advantage. Although both of them need to meet certain restrictions in the realization process, the combination of optical diffraction tomography \cite{ li2022transport,}, incoherent aperture scanning, and other technologies in the future will further improve the non-iterative and unconstrained performance of the optical system, and provide convenient and efficient tools for 3D imaging, digital refocusing, high-throughput digital pathology, and large-scale studies of biological cells and tissues.


%---------------------------------------------------------------------
\subsection{Structured illumination microscopy}
SIM applies periodic sine or cosine fringe structured light with a spatial frequency close to the diffraction limit through the sample, coupling the high-frequency information of the sample beyond the diffraction limit to the frequency region detectable by the imaging system\cite{chen2023superresolution}. For linear SIM, phase shift images in three directions of 0°, 60°, and 120° are usually required for reconstruction, which can double the cutoff frequency of the microscope. It is worth mentioning that SIM is suitable for imaging living samples in most cases, however, many intracellular processes are fast and SIM is only suitable for slowly moving structures. Moreover, traditional linear SIM is prone to noise artifacts, and phototoxicity needs to be considered under saturation conditions. Compared with SIM, FPM can increase the cut-off frequency by more than two times. At the same time, the FPM algorithm can be used to realize speckle structured light illumination fluorescence super-resolution imaging, which can increase the resolution of the linear region to four times\cite{chen2023superresolution,gustafsson2005nonlinear}.


%---------------------------------------------------------------------
\subsection{Phase contrast microscopy}
Since Fritz Zernike established the concept of phase contrast\cite{zernike1942phase}, the long ignored imaging information of optical phase has become of great importance, technologies such as Zernike phase contrast microscopy and differential interference contrast\cite{nomarski1955application} have made an impressive development. In 1984, combining the scanning microscopy imaging system with separate sensors, Hamilton and Sheppard proposed DPC\cite{hamilton1984differential} which took an intensity measurement resulting from the sampling phase and successfully realized the qualitative observation of the 3D information of the sample with better imaging contrast and robustness. On this foundation, Laura Waller and Lei Tian proposed a quantitative DPC imaging technology in 2015 with a microscope illuminated by a LED array\cite{tian2015quantitative}. By collecting four intensity images under the various semicircle illumination to do a subtraction between the two of them, the technology is supposed to get phase contrast images alongside the x, and y axes in the imaging plane, which can reach quantitative phase images after a step of deconvolution. Furthermore, there are single-shot methods for DPC with mult-wavelength\cite{liu2023single}.
Compared to the other phase retrieval methods, DPC performs better on high-speed imaging for its low complexity algorithm, less image capture, and high efficiency. But when comparing it with FPM, we found that DPC can only achieve a double resolution as it doesn’t take dark field images, which is lower than FPM. Apart from this, FPM enables researchers to obtain both intensity images and phase images, but for DPC, intensity images are not available. In addition, FPM has a better performance than DPC on aspects like extension of depth of field(DOF), high-throughput images, and aberration correction.  


\subsection{Coded ptchography}
The special feature of coded ptychography(CP) is its modulation plane, which can be set on the cover glass in front of the sensor. By etching micron-scale phase scatterers on the cover glass and printing sub-wavelength absorbers, the large-angle diffracted light waves can be redirected to smaller angles that can be detected by the pixel array below, and high-frequency information can be modulated into the system. Detection is performed in the low-frequency passband.
In coded ptychography, for thin coding surfaces, it can be assumed that diffracted waves at different incident angles remain unchanged during propagation; for thick coding surfaces, the full transmission matrix needs to be measured to characterize its modulation characteristics. Compared with other imaging methods, CP has the following characteristics\cite{wang2023optical}:



\begin{itemize}
    \item  Lensless high-resolution imaging is possible through the modulation effect of the coding surface;
    
    \item The FOV is large and the scanning step size is micron level, so the camera can scan at full frame rate;
    
    \item The traditional p probe beam will change in different experiments and for different samples. During the phase recovery process, it is usually necessary to jointly recover the probe beam and the object. When both the object and the probe beam contain slowly varying phase features of many $2\pi$  wraps, it is difficult to completely unambiguously separate the two from each other. By encoding the detection beam in hardware, CP can not only ensure the consistency of the detection beam in different experiments, but also quantitatively recover the slowly changing phase profiles of different samples;
    
    \item The traditional p overall acquisition process is a closed loop, requiring the mechanical stage to transmit a position feedback to the system, so a higher-precision motor is indispensable. The small distance between the object and the coded sensor allows the direct recovery of the object’s positional shift based on the raw diffraction measurements, so the system can be built with low-cost stepper motors.
\end{itemize}

\begin{figure}[H]
    \centering
    \includegraphics[width=13.8cm]{pictures/CP.png}
    \caption{\textbf{(a)}: Rapid whole slide imaging using the parallel coded ptychography platform. Reproduced from \cite{jiang2021resolution}. CC BY 3.0. (a1) The focus map generated by maximizing a focus metric post-measurement. (a2) The recovered whole slide image by coded ptychography. (a3) The ground truth image was captured using a regular light microscope. (a4) The difference between (a2) and (a3). \textbf{(b)}: High-throughput urinalysis based on the rotational coded ptychography platform built with a Blu-ray drive. Reproduced from \cite{jiang2022blood}. CC BY 3.0. \textbf{(c)}: Large-scale bacterial growth monitoring for rapid antimicrobial drug screening. Reproduced from \cite{jiang2022ptychographic}. CC BY 3.0. By imposing the temporal correlation constraint in coded ptychography, the imaging platform can achieve a centimeter-scale FOV, a half-pitch resolution of 488 nm, and a temporal resolution of 15-second intervals. Reproduced from \cite{jiang2022ptychographic}. CC BY 3.0. \textbf{(d)}:The recovered large-scale phase image of U87MG cell culture obtained by a lensless coded ptychography platform. Reproduced from \cite{jiang2020wide}. CC BY 3.0.}
    \label{fig:enter-label}
\end{figure}

Figure~\ref{fig:enter-label} shows the applications of CP in pathology, high-throughput screening, and microbial detection and detection respectively. It can be seen that CP can achieve large FOV and high-resolution imaging. Overall, there are four main differences between CP and Fourier pthography(FP):


\begin{itemize}
    \item  FP replaces the coding layer of CP with pupil aperture in Fourier space;
    
    \item  The light wave has undergone two Fourier transformations in the FP system, while CP is lensless, which is equivalent to the light wave propagating in free space;
    
    \item FP mainly depends on how the beam enters the sample, and CP depends on how the beam leaves the sample, so FP has certain requirements for the sample thickness, while CP can image 3D samples of any thickness;
    
    \item  Since the pupil aperture changes with different spatial positions of the imaging FOV during the FP process, usually if you want to obtain the best results, you need to model and restore the pupil.
\end{itemize}

%-------------------------------------------------------------------------------

\section{Technology}

%-------------------------------
\subsection{Research to improve FPM speed}
If the FP process is divided into two parts: image acquisition and image reconstruction, then the overall time of FP can be reduced by reducing the image acquisition time and improving the speed of the reconstruction algorithm. At present, the subsequent iterative reconstruction process is completed in the computer, and it is difficult to quantify its speed. Therefore, the current research focus on improving FPM speed is to reduce the sampling time. Since FP requires a certain overlap of spectra for the collected images, it will be simplified to the traditional phase retrieval process. Obviously, if the image overlap rate is high, more images need to be collected, which will require more sampling time; at the same time, if the overlap rate is too low, the sampling time can be greatly reduced, but the image recovery quality will inevitably be poor. Research on how to minimize FP sampling requirements has continued to develop over the years.

\begin{figure}[H]
    \centering
    \includegraphics[width=13.8cm]{pictures/speed.png}
    \caption{\textbf{(a1)-(a2)}: Input high-resolution intensity and phase profiles of the simulated complex sample. Reproduced from \cite{dong2014sparsely}. CC BY 3.0. \textbf{(b)-(d)}: FP reconstructions with different spectrum overlapping percentages in the Fourier domain. \textbf{(e)}: The RMS errors of the FP reconstructions versus the spectrum overlapping percentages. Reproduced from \cite{dong2014sparsely}. CC BY 3.0. \textbf{(f)}: Comparison between sequential FPM and source-coded FPM. Reproduced from \cite{dong2014sparsely}. CC BY 3.0. \textbf{(g)}: The experimental setup involves an LED array board, a cemented doublet condenser, an Olympus IX73 microscope with an Olympus UPlanSApo 10× (0.40 NA) objective lens, and a scientific CMOS camera. Reproduced from \cite{sun2018high}. CC BY 3.0. \textbf{(h)}:Hybrid illumination modes of ESA-FPM. Reproduced from \cite{fan2023efficient}. CC BY 3.0. }
    \label{fig:enter-label}
\end{figure}

In 2014, Zheng proposed Sparsely sampled FP and discussed the minimum overlap rate in the article. The team simulated situations with different overlap ratios, in which the incident wavelength was 632$nm$, the pixel size was 2.75$um$, and the objective lens NA was 0.08. The simulation used a 15×15 LED array to illuminate the sample at different angles. By adjusting the distance between the LED array and the sample, different spectral overlap ratios are achieved\cite{dong2014sparsely}. As shown in Figure~\ref{fig:enter-label}, (a1) and (a2) are the high-resolution intensity and phase images of the simulation input, 1(b)-1(d) are the intensity images reconstructed by FP under different overlap rates, 2(b) -2(d) is the phase diagram under different overlap ratios. In order to quantify the image quality of FP reconstruction under different overlap ratios, root-mean-square (RMS) errors (i.e., the difference between the ground truth and the recovered image) are defined to quantify them. Figure~\ref{fig:enter-label}(e) After multiple simulations The drawn line graph shows that there is not a simple linear relationship between the overlap rate and image recovery quality. At the same time, this image illustrates that successful FP reconstruction requires at least 35\% overlap. In the sparse sampling, FP subsequently proposed, one original pixel is divided into four sub-pixels for down-sampling. Only one of the four sub-pixels is updated in each iteration, and the others remain unchanged. At the same time, a down-sampling mask is used to solve the pixel Aliasing problem. By selectively updating pixel values in the spatial domain, this solution is able to get rid of the multiple exposure acquisition process in the original FP platform and greatly shorten the acquisition time. In the experiment of this scheme, USAF was used as the sample, the incident wavelength was 0.63$um$, the pixel size was 4.125$um$, the objective lens NA was 0.1, and a 15×15 LED array was used to illuminate the sample at different angles. The experiment showed that this scheme is completely feasible.


In 2015, Tian proposed a source-coded FPM technology for mixed lighting, the principle of which is shown in Figure~\ref{fig:enter-label}(f). In the lighting part, first captures four DPC images (top, bottom, left, and right half-circles) to cover the bright-field LEDs, and then uses random multiplexing with eight LEDs to fill in the dark-field Fourier space region. Secondly in the reconstruction part, use a linearly approximated phase solution based on DPC deconvolution as a close initial guess for spatial frequencies within the 2 NA bandwidth\cite{tian2015computational}. Through the above solution, the number of collected images can be reduced to 21, and the collection time only takes 0.8s. In the experiment, 21 images (taking 0.8s) and 173 images (taking 1min) were collected from a single frame of adult rat neural stem cells for reconstruction and recovery. The reconstruction results show that this method significantly reduces the number of images required and effectively improves the overall speed of FPM while ensuring a certain resolution.


In 2018, Sun proposed High-speed Fourier ptychographic microscopy based on programmable annular illuminations, using only four low-resolution images corresponding to oblique illumination to achieve high-speed imaging results for HeLa cells, the principle of which is shown in Figure~\ref{fig:enter-label}(g). For FPM, low-frequency phase information is difficult to recover. Only when the LED is accurately located at the edge of the objective lens in the frequency domain can the low-frequency phase information be correctly recovered. In order to ensure accurate phase reconstruction, this method first requires precise adjustment of the position of the LED array. Secondly, using a ring lighting scheme, by lighting only the LED elements on the ring, only 4-12 brightfield raw images are needed to achieve high-precision phase retrieval, significantly reducing data redundancy. An important reason why traditional FPM takes so long is that dark field images require a longer exposure time (generally each dark field image requires 30$ms$ exposure) to ensure a certain signal-to-noise ratio \cite{sun2018high}, while this method only uses bright field Image reduces exposure time to 10$ms$ per raw image.Under the above premise, the team only needs to light up 4 LEDs at the fastest, which can reduce the collection time to 0.04s. In the same area, but in comparison, AIFPM can bypass the influence and limitations of the camera pixel size and resolves the two closely spaced features with distance of 655 $nm$.


In 2022, Zuo also proposed an efficient synthetic aperture for phaseless Fourier ptychographic microscopy with hybrid coherent and incoherent illumination (ESA-FPM). Figure~\ref{fig:enter-label}(h) shows the hybrid lighting process of ESA-FPM.During the image acquisition process, when collecting bright field images, all LEDs are turned on at the same time, and a single intensity image is collected under incoherent lighting conditions; when collecting dark field images, sparse sampling is based on central symmetry to ensure improved resolution. The imaging resolution limited by coherent diffraction is achieved with only 7 original images. Compared with traditional FPM, the amount of data required is only 1.6\%. As lighting methods change, iteration strategies are bound to change as well.During each iteration, the sub-aperture spectrum corresponding to the bright-field illumination is extracted and its amplitude is updated by the bright-field measurement. The spectrum filtering function is defined as OTF instead of CTF to conform to incoherent imaging. Without sub-aperture scanning, one update fills the spectrum with 2× the coherent diffraction bandwidth of the objective lens. For dark-field illumination, the same dark-field intensity measurement is filtered by CTF to sequentially update two sub-apertures (sub-aperture 1 and sub-aperture 2) of the centrosymmetric distribution\cite{fan2023efficient}. At the same time, in order to avoid falling into a local minimum, an adaptive compensation strategy is added to update the weights. Figure ~\ref{fig:enter-label}(i) and Figure ~\ref{fig:enter-label}(j) show the experimental result of ESA-FPM on pathological sample of Lymph node metastatic in squamous cell carcinoma. It can be seen that EAS-FPM produces visualization results with significantly improved resolution, such as blood cells and lymphocytes. Cells can be clearly distinguished.

%---------------------------------------------------------------------
\subsection{Three-dimensional imaging}
In the biomedical areas, high-resolution 3D imaging attracts wide attention due to more information contained in the cells or tissues. However, it is extremely difficult for the traditional microscope to reconstruct thick samples in 3D. However, for FPM, due to the long DOF \cite{zheng2013wide} and other characteristics, there are three main methods to realize 3D imaging, digital refocusing, multi-slice, and Optical coherence tomography. 

It is necessary to perform mechanical scanning to gain the 3D volume imaging sample for the traditional microscope. However, there are defocused artifacts due to the interference of the upper and lower planes outside the DOF. And the DOF is defined by $DOF = \frac{\lambda }{{N{A^2}}} + \frac{\lambda }{{M\times NA}}e$, where is the minimum resolvable distance of detector and M stands for magnification \cite{pan2020high}. Luckily, FPM can achieve digital refocusing, leveraging the feature of coherent light. FPM can “digitally refocus” images by numerically zeroing out the defocus aberration Digital refocusing refers to the algorithm focusing method according to the characteristics of the data after the completion of imaging data acquisition. Zheng and Bian et al. multiply the phase term which is related to defocusing distance into the pupil function. Then this pupil function as follows:

\begin{equation}
P'({k_x},{k_y}) = P({k_x},{k_y}){e^{jz\sqrt {{k_0}^2 - {k_x}^2 - {k_y}^2} }},{k_x}^2 + {k_y}^2 < {({k_0}NA)^2}, 
\end{equation}
where $z$ denotes the defocus amount. 

It is performed by determining and removing the defocus aberration during the iterative phase retrieval process to extend the DOF. Using a ×2 apochromatic objective lens (Plan APO, 0.08 NA, Olympus), the FPM can digitally refocus to extend the depth of field to 300 $\upmu$m \cite{zheng2013wide}. This operation allows us to eliminate the defocus aberration in each pupil function at each illumination angle. However, this algorithm requires knowledge of the out-of-focus distance $z$ as a priori information. Otherwise, it often consumes a lot of computational time to search for a suitable out-of-focus distance.  Furthermore, Zhou et al. demonstrated that the process of refocusing in FPM cannot be separated from the iterative phase retrieval procedure without prior information\cite{zhou2022analysis}. Zhang et al. proposed a fast digital refocusing and DOF extended FPM strategy by taking advantage of image lateral shift caused by sample defocusing and varied-angle illuminations \cite{zhang2021fast}. The degree of lateral shift is directly related to the amount of defocusing and the tangent of the illumination angle. Instead of using a time-consuming optimization search to find the best defocus distance, this method allows for precise and quick determination of the defocus distance for each subsection of the sample by calculating the relative lateral shifts corresponding to different oblique illuminations. The effectiveness of this approach in achieving rapid digital refocusing and extending the DOF was confirmed through practical experiments using a USAF chart (Figure \ref{3D_1}(a)). Zheng et al. have employed coded-detection methods to separate the sample-focusing step from the iterative phase retrieval process. By obtaining the retrieved object exit wavefront, it becomes possible to propagate it to various axial positions and employ a focus metric to determine the most suitable focal position. For stained samples, focus metrics based on intensity can be applied, whereas for unstained samples, focus metrics based on phase are more appropriate \cite {guo2021deep, guo2020openwsi, jiang2022high}. However, the maximum length of the depth-of-field extension achieved by digital refocusing does not exceed the coherence length. The system satisfies the approximation of a two-dimensional thin object only if the thickness of the sample satisfies certain conditions\cite{ lee2013synthetic}. Similar to stacked diffraction imaging, FPM requires that the thickness of the observed sample generally does not exceed 10 $\upmu$m, which severely limits the application of FPM in 3D microscopy and imaging. Imaging \cite{ liu2009influence}. Zheng et al. introduced a new imaging technique called ptychographic structured modulation for super-resolution microscopy \cite{song2019super}. They use a thin diffuser to manipulate light from the sample, allowing high-resolution details to be captured. Ptychographic structured modulation overcomes the diffraction limit, handles thin samples better, and provides accurate complex object contrast. The digital propagation of the recovered complex wavefront to two different layers. Additionally, they employ an aperture-scanning FP setup to retrieve the complex hologram of extended objects \cite{dong2014aperture}. This reconstructed hologram is digitally propagated through various planes along the optical axis to investigate the 3D structure of the object in figure \ref{3D_1}(c). 

In view of the limitation of sample thickness, many studies follow the example of real-space ptychography imaging and combine a Multi-slice model with FPM technology \cite{maiden2009improved, maiden2012ptychographic, tian20153d, pan2016experimental, li2015separation}. Lei Tian et al. introduced the concept of multi-slice modeling in FPM \cite{tian20153d}. This innovative method enables the reconstruction of 3D sample information at different depths along the optical axis, effectively surpassing the limitations of traditional microscopes. In this multi-slice approach, the sample is divided into thin slices, and wave fields pass through each of them. However, as the number of slices increases, the computational workload grows significantly, impacting computational efficiency. Moreover, achieving high reconstruction accuracy, particularly when using high NA objectives, can be challenging due to light scattering in various directions, including sideways and backward. Shwetadwip Chowdhury et al. implement a new 3D refractive index (RI) microscopy technique that utilizes a computational multi-slice beam propagation method to invert the optical scattering process and reconstruct high-resolution (NA > 1.0) 3D RI distributions of multiple-scattering samples. The method acquires intensity-only measurements from different illumination angles and then solves a nonlinear optimization problem to recover the sample’s 3D RI distribution. And it also is demonstrated by the reconstruction of samples with varying amounts of multiple-scattering A 3D RI of 3T3 fibroblast cell is shown in Figure \ref{3D_1}(b).

\begin{figure}[H]
\includegraphics[width=13.7cm]{pictures/3D_1.png}
\caption{\textbf{(a)}: High-resolution reconstruction results of conventional FPM and the Zhang et al. proposed method with different defocus distances. Reproduced from \cite{zhang2021fast}. CC BY 3.0.
\textbf{(b)}: The reconstruction of samples with varying amounts of multiple-scattering A 3D refractive index (RI) of 3T3 fibroblast cell. Reproduced from \cite{tian20153d}. CC BY 3.0.
\textbf{(c)}: 3D holographic refocusing using the aperture-scanning FP.The recovered sections at (c1) z = −1300 $\upmu$m, (c2) z = −1000 $\upmu$m, (c3) z = −700 $\upmu$m, and (c4) z = −400 $\upmu$m. Reproduced from \cite{dong2014aperture}. CC BY 3.0.

\label{3D_1}}
\end{figure}


Unfortunately, refocusing does not remove light from areas above and below the plane of interest. However, the multi-slice method does not directly account for backscattered light. Its projection approximation also assumes the lateral divergence of the optical field gradient at each slice is zero. This longstanding problem has inspired a number of solutions. Roarke et al. performed diffraction tomography (DT) using standard intensity images captured under variable LED illumination from an array source, termed Fourier ptychographic tomography (FPT) \cite{horstmeyer2016diffraction}. Diffraction tomography, initially proposed by Emil Wolf \cite{wolf1969three},  combines the theory of X-ray tomography and holography to estimate the 3D RI distribution of an object under weak scattering (single scattering) approximation. Roarke et al. employed diffraction tomography, utilizing intensity measurements acquired with a standard microscope and an LED illuminator. This system achieves a lateral resolution of about 400 nm, operating at the Nyquist–Shannon sampling limit. Regarding axial resolution, it reaches 3.7 $\upmu$m at the same sampling limit. Their experiments extended up to a maximum axial depth of 110 $\upmu$m along the z-axis. Remarkably, they successfully demonstrated quantitative measurements of the complex refractive index across various thick specimens featuring contiguous structures. Michael Chen and Laura Waller et al. introduced a precise and computationally efficient 3D scattering model known as the multi-layer Born model. They harnessed this model to recover the 3D RI of thick biological specimens \cite{chen2020multi}. To implement MLB effectively, they integrated it into a phase tomography framework, relying solely on intensity-only images acquired using the FPM. In a separate study, Sandro et al. applied FPT to coherent anti-Stokes Raman scattering imaging. Their work demonstrated that complex third-order susceptibilities can be reconstructed in 3D using synthetic data and a limited number of widefield coherent anti-Stokes Raman scattering images. Zuo et al. developed Fourier ptychographic diffraction tomography (FPDT), an approach for generating high-resolution 3D RI images across substantial volumes. FPDT accomplishes this by utilizing low-NA intensity measurements and incorporating high-angle dark-field illumination \cite{ zuo2020wide}. The results are truly impressive, featuring a wide FOV measuring 10 × FOV of 1.77 $mm^2$, with exceptional 390 nm lateral resolution, 899 nm axial resolution, and a depth of focus of approximately 20 $\upmu$m, as shown in Figure \ref{3D_2}(a). Li et al. have introduced an innovative label-free 3D microscopy technique termed transport of intensity diffraction tomography with non-interferometric synthetic aperture" (TIDT-NSA) \cite{li2022transport}. This approach allows for the retrieval of the 3D RI distribution of biological specimens from 3D intensity-only measurements acquired at various illumination angles, enabling incoherent-diffraction-limited quantitative 3D phase-contrast imaging. Utilizing an off-the-shelf bright-field microscope equipped with a programmable LED illumination source, TIDT-NSA achieves an impressive imaging resolution of 206 nm laterally and 520 nm axially when using a high NA oil immersion objective. One notable advantage of TIDT-NSA is its ability to eliminate the need for a matched illumination condition (analyticity condition) typically required in 2D Kramers-Kronig relations. Instead, it leverages 3D intensity transport, which provides direct access to the object's frequency content within the generalized aperture. Furthermore, a unified transfer function theory of 3D image formation has been derived, establishing a link between the 3D object function (scattering potential) and the 3D intensity distribution under first-order Born/Rytov approximations. This enables direct non-interferometric 3D synthetic aperture imaging in the Fourier domain, as shown in Figure \ref{3D_2}(b). Building upon Li's research, Zhou et al. have introduced transport-of-intensity Fourier ptychographic diffraction tomography" (TI-FPDT) as a solution to address challenging issues in microscopy \cite{zhou2022transport}. TI-FPDT combines ptychographic angular diversity with additional "transport of intensity" measurements. This approach takes advantage of defocused phase contrast to circumvent the stringent requirement on illumination NA imposed by the matched illumination condition. Consequently, TI-FPDT effectively mitigates issues related to reconstruction quality deterioration and refractive index underestimation seen in conventional FPDT. These improvements are evident in high-resolution tomographic imaging, like the imaging of USAF targets, as shown in Figure \ref{3D_2}(c). Habib et al present a parallel implementation of a synthetic aperture in TIDT (PSA-TIDT) using annular illumination \cite{ ullah2023parallel}. This matched annular illumination setup yields a mirror-symmetric 3D optical transfer function, signifying the analyticity in the upper half-plane of the complex phase function. This unique feature enables the recovery of the 3D RI from a single-intensity stack. To validate the effectiveness of PSA-TIDT, they conducted rigorous experimental tests. These experiments included high-resolution tomographic imaging, resulting in the reconstruction of the 3D RI rendering of Henrietta Lacks (HeLa) cells.
\begin{figure}[H]
\includegraphics[width=13.7cm]{pictures/3D_2.png}
\caption{\textbf{(a)}: Five 2D slices of the RI distribution at different depths recovered using FPDT without (left) and with dark-field (right) imaging. Reproduced from \cite{ zuo2020wide}. CC BY 3.0.
\textbf{(b)}: Full FOV and different tomogram ROIs at different positions and axial planes to illustrate the recovered RI slice results of C. elegans and 3D RI rendering of C. elegans worm over a volume of 180 $\upmu$m × 180 $\upmu$m × 115 $\upmu$m. Reproduced from \cite{li2022transport}. CC BY 3.0.
\textbf{(c)}: In-focus and defocused intensity images of phase resolution target at vertical coherent illumination. Reproduced from \cite{zhou2022transport}. CC BY 3.0.
\label{3D_2}}
\end{figure}

%---------------------------------------------------------------------
\subsection{Full-color acquisition}
Given the sensitivity of human vision to color information and our innate tendency to classify objects based on their hue, it is common practice to stain biological samples for enhanced recognition and categorization. Therefore, digital pathology places significant emphasis on recovering and acquiring accurate color representation of stained histopathological samples. FPM offers a significant advantage in generating large FOV, high-resolution, and full-color whole slide images without the need for mechanical scanning. This is achieved through five different methods, which ensure full-color image acquisition.

FPM can accommodate both monochrome and chromatic cameras. With a monochrome camera, FPM can retrieve high-resolution images at three different wavelengths (red, green, and blue) and combine them into a single, high-resolution, full-color image. This process involves sequentially illuminating the slide with red, green, and blue light to obtain the color information, which can be easily achieved using a programmable R/G/B LED array. While there are no additional requirements for overlapping ratio or system environment compared to traditional FPM, it's important to note that the restored full-color images may exhibit coherent artifacts caused by dust particles on the slide or lens. Moreover, color reconstruction can be time-consuming, and intense calibration of the three wavelengths is required. Similar to monochrome cameras, chromatic cameras with Bayer filters do not have additional limitations on overlap ratio or system environment. However, while they offer the advantage of separating three primary channels more efficiently, their larger pixel size often results in lower photographic efficiency. Furtherly, correcting for color leakage remains an important consideration\cite{zhou2017fourier}.

Significant efforts have been devoted to developing wavelength-multiplexed FPM that utilizes a monochrome camera and simultaneous multi-wavelength illumination\cite{dong2014spectral,zhou2017fourier,pan2016incoherent,sun2016sampling,wang2017color}. While this approach appears promising, reducing acquisition time by approximately two-thirds, it requires an overlap ratio of nearly three times that of traditional FPM\cite{sun2016sampling}. Moreover, the need for sophisticated algorithms to address decoupling issues increases system complexity, making error calibration challenging. Additionally, acquiring three separate low-resolution images in different wavelengths is necessary to prevent converging to similar gray values, further complicating the process\cite{sun2016sampling,wang2017color}.

It is also possible to apply a deep learning technique to perform unsupervised image-to-image translation of FPM reconstructions, thereby enhancing image quality and color accuracy while decreasing artifacts caused by coherent illumination\cite{zhu2017unpaired,wang2020virtual}. Specially, a cycle-consistent adversarial network with multiscale structure similarity loss is employed and trained using two sets of unpaired images. Notably, the network's output closely aligns with the ground truth intensity, and the overall image quality surpasses that of the FPM color image obtained through sequential red, green, and blue illuminations. Also, other than reducing the FPM coherent artifacts, this data-driven approach can shorten the acquisition time of FPM by 67\%.

Instead of employing a neural network for image translation, two novel approaches termed color transfer FPM (CFPM) and color-transfer filtering FPM (CFFPM) are proposed as an alternative to virtually stain an FPM image\cite{gao2021high,chen2022rapid}. Compared with the previous CFPM, CFFPM replaces the original histogram matching process with a combination of block processing and trilateral spatial filtering and solves the double-coloring problem as well as improves both the precision and speed of color transfer. CFFPM can perform accurate and fast color transfer for various specimens and for some cases, which can also outperform the sequential conventional methods because of the coherent artifacts introduced by dust particles.

%---------------------------------------------------------------------
\subsection{Deep Learning}

\begin{figure}[H]
\includegraphics[width=13.7cm]{pictures/DL_1.png}
\caption{Frameworks and reconstruction of deep learning methods working on FPM. \textbf{(a1)}: The deep learning framework of inferring HR images from LR images. \textbf{(a2)}:Details of the CNN structure in a1. \textbf{(a3)}: Frames of the reconstructed high-SBP phase images observing significant morphological changes over a  course of 4 hours. Reproduced from \cite{thanh2018deep}. CC BY 3.0. \textbf{(b1-b2)}: Structure and reconstruction comparison of deep learning framework based on a physical model named SwinIR. Reproduced from \cite{wang2022fourier}. CC BY 3.0.
\label{dp}}
\end{figure}

In recent years, applying deep learning methods to the reconstruction of FPM has become a common trend as these methods have been demonstrated to be powerful tools for solving data problems including the phase retrieval not only in FPM reconstruction\cite{zheng2021concept} but also in all QPI technologies\cite{qayyum2022untrained}. Three possible research directions divided by their roles in the FPM to improve the performance of the technology have been explored.
The first group worked on the research point that aims to infer the high resolution images from the easily obtained low resolution intensity or phase images, entirely considering the FPM reconstruction as a data proposal problem\cite{shamshad2019deep,xue2019reliable,kappeler2017ptychnet,boominathan2018phase,zhang2019fourier,thanh2018deep}. Actually, data-driven image reconstruction techniques based on deep learning, have gained tremendous success in solving complex inverse problems. Figure~\ref{dp}(a) shows the reconstruction results of a deep learning approach over an imaging course of 4 hours and explicitly shows the significant morphological changes during the process.
The second group sought the incorporation of the physical model and the deep learning framework, which made the whole model more interpretable\cite{kellman2019physics,muthumbi2019learned,cheng2019illumination,kellman2019data,horstmeyer2017convolutional}. Such physical models adopt modular design and feature fusion to complete the image reconstruction. Compared with the classical traditional method, this method greatly saves running time and speed and has good performance in the evaluation index, which provides a new improvement idea for future FPM reconstruction. 
The third group tried to model the forward imaging using a neural network and perform optimization via a network training process, which utilizes a forward pass to model the real imaging process of the actual FPM reconstruction process\cite{wang2020virtual,jiang2018solving,zhang2019pgnn,sun2019neural,zhang2020forward}. Figure~\ref{dp}(b) compared the reconstruction results of intensity and phase images between a forward imaging neural network and the traditional algorithms and other neural networks. Fige~\ref{dp}(c-e) shows the reconstruction results of another forward imaging neural network named FINN-P  with its intensity images, phase images, and pupil recovery. 
All these groups have obtained excellent progress recently. For the first group, different neural networks have been proposed. Researchers utilized the DeUNet to come across the Cross-level Channel Attention Network which combined the coding method for denoising in measurement\cite{zhang2021cross}. Residual Transfer Network was used to overcome the gradient explosion\cite{wang2022fourier}, making the feature information more complete and efficient, and the incremental up-sampling reconstruction network has higher image quality, lower computational complexity, and shorter running time. As a combination with the neural network and the previous phase retrieval algorithms, LWGNet tried to combine the Wirtinger gradients with a neural network and enhanced the gradient images by convolution modules which may significantly bring down the cost of FPM imaging setup\cite{saha2022lwgnet}. A double-flow convolution neural network which separates the network data flow into two branches has successfully reduced the processing time from 167.5 to 0.1125 second with fine generalizability and little dependence on the morphological features of samples\cite{sun2021double}. 

\begin{figure}[H]
\includegraphics[width=13.7cm]{pictures/DL_2.png}
\caption{Frameworks and reconstruction of deep learning methods working on FPM. \textbf{(a1)}:Overall workflow of FINN-P method. \textbf{(a2)}: Reconstruction results in an enlargement of FINN-P and the amplitude and phase of the pupil function recovered through FINN-P. Reproduced from \cite{sun2019neural}. CC BY 3.0. \textbf{(b1-b2)}: Structure and reconstruction results of deep learning framework based on deep image prior. Reproduced from \cite{chen2022fourier}. CC BY 3.0.
\label{dp}}
\end{figure}

For the second group, new physical models like SwinIR have been introduced into FPM for completing the reconstruction of images by introducing modular design and feature fusion\cite{wang2022fourier}. Through a series of simulation experiments on ideal images and real images, it is proved that this physical model has better reconstruction quality than the comparison algorithm. A similar idea of feature fusion was also demonstrated in the Residual Hybrid Attention Network\cite{li2023fourier}, which successfully simulated the network noises. Another physics-inspired model with deep learning framework was also proved to be efficient with a low overlap condition as it reached a super-resolution factor of γ=5 with only 37 images\cite{bouchama2023physics}. A neural network with a physics-based channel attention was also used for FPM reconstruction to enable the adaptive correction of the spatial distribution of LED intensity\cite{zhang2022physics}.
For the third group, setting the real part and imaginary part of the object as well as the different and irregular positional deviations of each aperture as the weights of the convolutional layer has been proven to be effective in accurately finding the aperture position and improving the
reconstruction quality\cite{zhao2021neural}.
Aside from these previous research points, deep image prior was also proposed into the FPM. As a deep learning method, it is able to provide a starting point, a prior, which allows a great reduction of the artifacts, especially for an illumination array of limited size\cite{guzzi2021deep}. Unlike the traditional training type of deep neural network that requires a large labeled dataset, an FPM with untrained deep neural network priors does not require training and instead outputs the high-resolution image by optimizing the parameters of neural networks to fit the experimentally measured low-resolution images\cite{chen2022fourier}. With the application of a priori information, deep neural networks can evidently reduce the requirement for practical applications of FPM\cite{yang2022fourier}.

%--------------------------------------------------------------------
%%%%%%%%%%%%%%%%%%%%%%%%%%%%%%%%%%%%%%%%%%
\section{Applications}
\subsection{Digital Pathology}

Digital pathology, regarded as a bridge between basic medicine and clinical diagnosis\cite{burger1991pathology}, has absorbed the latest achievements in whole slide imaging systems for their distinct and efficient observation of the changes in function, metabolism, and morphological structure of organisms. In this regard, FPM demonstrates many advantages and a promising future for digital pathology as a new generation of high-efficiency and advanced imaging modes.

Conventional microscopes have been suffering from the trade-off between FOV and spatial resolution, which in contrast can only achieve high resolution images through high-NA objectives and then stitch them together in the spatial domain to acquire a large field-of-view image with high resolution, inevitably leading to image artifacts and low efficiency, as well as a limited depth of field coming along with the high-NA objective\cite{zheng2013wide}. Compared to the conventional imaging modes, FPM can simultaneously acquire images of histology sections with both a large FOV and high resolution without any manual post stitches. 

Meanwhile, due to the small depth of field of high-NA objectives, constant refocusing is required when the FOV changes to keep the sample in focus, which is time-consuming and cumbersome. FPM allows for digital refocusing during the reconstruction process to address this problem, bringing every small segment of the image into focus\cite{claveau2020digital} and recovering the imaging system’s pupil function to further improve the quality of images\cite{ou2014embedded,song2019full}. As shown in Figure~\ref{pathodoly}(a-d), unlike other imaging systems, the digital refocusing capability of FPM allows for imaging of cells at different focal planes to be achieved on the same plane, despite surface unevenness, thereby facilitating the identification and enumeration of circulating tumor cells (CTCs). Additionally, the large FOV provided by FPM avoids the issues of CTCs failing to be counted or a single event being counted more than once during the stitching process\cite{williams2014fourier}. Similarly, a large FOV of cell analysis can also be performed on the FP-recovered high-content images. As demonstrated by the FPM results of thyroid fine needle aspiration cytology samples in Figure~\ref{pathodoly}(e), it is evident that all nonoverlapping cells in the diagnostic cluster can be seen in sharp focus, regardless of the thickness of the samples\cite{liang2022all}.

Another important application of FPM in digital pathology is to extract the scattering characteristics of tissue samples from the recovered phase information. This scattering characteristic is directly related to the RI of the tissue and helps distinguish healthy cells from cancerous cells in digital pathology diagnosis\cite{horstmeyer2015digital}. As shown in the zoomed-in views of Figure~\ref{pathodoly}(f), the recovered phase can be used to obtain the local scattering and reduce the scattering coefficients of the specimen. Also, FPM's captured quantitative images with high contrast are an ideal method for non-invasively monitoring biological samples and analysis. As shown in Figure~\ref{pathodoly}(g), it is evident that the high-contrast areas corresponding to the nuclei in the H&E-stained renal tissue slide are distinguishable as well in the stain-free FPM image\cite{valentino2023beyond}. At the same time, the unstained kidney tissues section images provided by FPM can also obtain quantitative phase information while avoiding the influence of staining color and intensity on imaging effects which comes along with the markers. While taking cell segmentation and counting of several specific types into consideration, FPM can provide higher overall image quality compared to fluorescent imaging and DPC, with higher contrast and clear visible fine details\cite{wakefield2022cellular}.


\begin{figure}[H]
\includegraphics[width=13.7cm]{pictures/digital .png}
\caption{\textbf{(a-d)}: Full FOV color image of the entire microfilter containing captured tumor cells. Magnified regions indicate the level of detail of cell morphology that can be seen. Reproduced from \cite{williams2014fourier}. CC BY 3.0. \textbf{(e)}: Comparison between a single plane of focus image with an all-in-focus FPM image. Single-focal-plane color image reconstructed from FPM. Note the large areas of out of focus image in the diagnostic cluster that results from the thickness of the preparation. All-in-focus color image reconstructed from FPM. Reproduced from \cite{liang2022all}. CC BY 3.0. \textbf{(f)}: The phase map shows the optical path length delays introduced by the specimen and the reduced scattering coefficient map quantifies how much light has been scattered by the specimen. Reproduced from \cite{horstmeyer2015digital} by permission from Elsevier: Computerized Medical Imaging and Graphics © 2015. \textbf{(g)}: Unstained and H&E-stained images comparison. Reproduced from \cite{valentino2023beyond}. CC BY 3.0. \label{pathodoly}}
\end{figure}  

%--------------------------------------------------------------------------
\subsection{Drug screening}

As shown in Figure~\ref{drug1}(a), drug screening has gone through five different stages of development: physiology-based high-throughput screening (HTS), chemistry-based HTS, high-content screening (HCS), high-screening imaging (HCI), high-throughout HCI (HT-HCI). HTS is a well-established process for lead discovery in Pharma and Biotech companies\cite{bleicher2003hit} which is shown in Figure~\ref{drug1}(b), and it comprises the screening of large chemical libraries for activity against biological targets via the use of miniaturized assays and large-scale data analysis\cite{mayr2009novel}, such as flow cytometry which enables the simultaneous quantitative analysis in individual cells\cite{edwards2004flow}. However, as shown in Figure~\ref{drug1}(c), this conventional method has been suffering from time, cost, and quality and has been difficult to adapt to the research of related therapeutic drugs for polygenic diseases and viral infections\cite{mayr2009novel}. Compared to HTS, HCS combines the efficiency of high-throughput techniques with the ability of cellular imaging to collect quantitative data from complex biological systems\cite{zanella2010high, boutros2015microscopy, lang2006cellular}. Among them, FPM-based screening systems enable simultaneous imaging and analysis of the morphology of cells, greatly enhancing the experimental scale and efficiency of drug screening. 

\begin{figure}[H]
\includegraphics[width=13.7cm]{pictures/drug1.png}
\caption{\textbf{(a)}: Five stage of massive drug screening. \textbf{(b)}: Progress for lead discovery and corresponding stage-by-stage quality assessment to reduce costly late-stage attrition. Reproduced from \cite{bleicher2003hit} by permission from Springer Nature: Nature Reviews Drug Discovery © 2003. \textbf{(c)}: The optimization process for successful HTS and the key success factors for modern lead discovery via HTS, namely time, costs, and quality. Reproduced from \cite{mayr2009novel} by permission from Elsevier: Current Opinion in Pharmacology © 2009. \label{drug1}}
\end{figure}  

Currently, cell-based high-throughput screening systems based on commercial porous plates can only provide limited and rough descriptions of the samples' properties. Given the high-throughput advantage of FPM which can provide rich information about cell cultures such as morphology, integrity, and viability, a 6-well and 96-well cell culture imaging and drug screening system based on parallel FPM have been sequentially developed and reported \cite{kim2016incubator, chan2019parallel, pan2019situ}, named 6 Eyes and 96 Eyes, respectively. As shown in Figure~\ref{drug2}, it demonstrates a parallel microscopy system that can simultaneously image all wells on a 96-well plate via FP and can perform dual-channel fluorescence imaging at the native resolution of the 96 objectives. 



Although excellent methods exist to image quantitatively complex processes such as protein–protein interactions by fluorescence resonance energy transfer\cite{erfle2004sirna, wouters2001imaging} or the dynamics of the turnover of fluorescent proteins on cellular organelles \cite{lippincott2001studying, rabut2004mapping, forster2006secretory}, their application for object recognition algorithms for ‘real time’ image analysis in a high-throughput microscopy still needs to be developed\cite{starkuviene2007potential}. The combination of increasingly advanced artificial intelligence and high-throughput FPM not only successfully meets this requirement but also promotes the development of drug screening. The number of white blood cells (WBCs) is a valuable indicator for diagnosing or predicting various diseases\cite{yarnell1991fibrinogen, kannel1992white, grimm1985prognostic, barron2000association, brown2001white, twig2012white}. An automatic counting algorithm was developed to accurately distinguish WBCs from FPM's recovered images with 95\% accuracy \cite{chung2015counting}. Moreover, machine learning (random forest) and deep learning (VGG16) models can be used to diagnose the infection status of thousands of red blood cells within a single FOV, achieving 91\% and 98\% specificity, respectively\cite{akcakir2022automated}.



\begin{figure}[H]
\includegraphics[width=13.7cm]{pictures/drug2.png}
\caption{ \textbf{(a)}: General hardware. Individual plates are loaded from the front. Reproduced from \cite{pan2019situ}. CC BY 3.0. \textbf{(b)}: The imaging module consists of 96 repeating units of compact miniaturized microscopes packed in a 9 mm × 9 mm × 81 mm space, where they all share the same light source. Reproduced from \cite{pan2019situ}. CC BY 3.0. \textbf{(c)}: Final performance of the 96 Eyes with dual-channel fluorescence imaging. Reproduced from \cite{pan2019situ}. CC BY 3.0. \label{drug2}}
\end{figure}  


%----------------------------------------------------------------------
\subsection{Label Free}
In vitro microscopy is crucial for studying physiological phenomena in cells. For many applications, such as drug discovery \cite{lang2006cellular}, cancer cell biology \cite{boyd1995some}, and stem cell research \cite{costa2011continuous}, the goal is to identify and isolate events of interest. Free from the adverse effects of staining reagents on cell viability and signaling in existing imaging techniques \cite{zheng2011epetri, greenbaum2012imaging, greenbaum2014wide, luo2015synthetic, goda2009serial}, high-throughput FPM has been demonstrated for long-term label-free observation and quantitative analysis of large cell populations without compromising the spatial and temporal resolution, and shown great promises in important applications in drug discovery, personalized genomics, cancer diagnosis, and drug development \cite{lukosz1966optical, zheng2013wide, candes2008introduction}.

In contrast to fixed slides, live samples are continuously evolving at various spatial and temporal scales. Faster capture times would not only improve the imaging speed, but also allow studies of live samples, where motion artifacts degrade results. A new source coding scheme was proposed and achieved 0.8 NA resolution across a 4× FOV with subsecond capture times \cite{tian2015computational}. As shown in Figure~\ref{labelfree}(a), a HeLa cell is undergoing mitosis and dividing into four cells and the whole process stays in focus across the FOV where subcellular features are visible and the dynamics of actin filament formation can be tracked over time. Also, both fast-scale dynamics and slow-scale evolution of adult rat neural stem cells with details at both the subcellular level and across the entire cell population can be seen in Figure~\ref{labelfree}(b). Moreover, in order to address the time-varying aberration and focus drifts in long-term live-cell imaging, a computational QPI method based on annular illumination FPM was reported, in which the annular matched illumination configuration. As shown in Figure~\ref{labelfree}(c), the mother cell underwent three mitoses across ∼27 hours and was eventually divided into four individual daughter cells with high imaging quality\cite{shu2022adaptive}, which reveals the capacity of FPM-based approach for correcting temporally varied aberrations and securing the imaging performance for a long-term longitudinal study \cite{chung2016wide, kamal2018situ, konda2021multi, aidukas2019low, shen2019computational}.


\begin{figure}[H]
\includegraphics[width=13.7cm]{pictures/labelfree}
\caption{\textbf{(a)}: A zoom-in of one small area of confluent cells in which one cell is dividing into multiple cells. Reproduced from \cite{tian2015computational}. CC BY 3.0. \textbf{(b)}: A zoom-in of one small area. Top: successive frames at the maximum frame rate (1.25 Hz). Bottom: sample frames across the longer time lapse (4.5 h at 1 min intervals). Reproduced from \cite{tian2015computational}. CC BY 3.0. \textbf{(c)}: The 3-generation cell division process from 8h 6min to 35h 6min. Reproduced from \cite{shu2022adaptive}. CC BY 3.0.   \label{labelfree}}
\end{figure}

%%%%%%%%%%%%%%%%%%%%%%%%%%%%%%%%%%%%%%%%%%%%%%%%%%%%%%%%%%%%%%%%%%%%%%%
\section{Outlook}
Although FPM demonstrates its versatility as an imaging technique, there has still been imperfections of FPM to date for new scientific discovery. Here we explore the exciting possibilities and future directions for FPM in technological domains.

\begin{itemize}

     \item \textbf{Algorithms}: New algorithms for FPM continuously emerged in recent years with better speed of convergence and robustness. Wu et.al proposed an aberration correction method, by applying an adaptive modulation factor into the FPM reconstruction framework, this algorithm performed better robustness and convergence for eliminating hybrid aberrations\cite{wu2023adaptive}. In the quest for higher resolution, FPM algorithms are increasingly employing advanced reconstruction techniques, such as sophisticated phase retrieval algorithms and non-linear optimization approaches. Deep learning methods are also being integrated into FPM algorithms to enhance noise reduction and adapt to challenging imaging conditions.
     
    \item \textbf{Speed and efficiency}: Pioneering swifter and more efficient FPM algorithms alongside optimized hardware configurations could render the technique significantly more viable across a spectrum of applications, including real-time imaging. Notably, the development of real-time imaging systems that incorporate deep learning can revolutionize FPM applications. These systems can rapidly analyze acquired data and make real-time decisions based on FPM-generated information, opening up new possibilities in dynamic and time-sensitive scenarios.
    
    \item \textbf{Polarization}: The properties of polarization add another dimension to FPM. Polarization-sensitive FPM can provide enhanced contrast and resolution, particularly in biological and materials science applications, where the orientation and anisotropic properties of samples are of critical importance. Although instances can be found within the broader field of ptychography, there has not been a documented instance of polarization imaging utilizing FPM, although instances can be found within the broader field of ptychography.

      \item \textbf{3D imaging}: FPM is embarking on a promising journey towards accurate 3D reconstructions of samples without the need for scanning. This approach opens up new horizons for researchers, enabling them to explore complex structures in a non-destructive and time-efficient manner. Although numerous endeavors have already expanded the frontiers of FPM to reconstruct 3D images from thick samples, as previously mentioned, further endeavors are essential to comprehensively grasp and model the intricate intricacies of multiple scattering.


     \item \textbf{Multimodal imaging}: The fusion of FPM with other imaging techniques, such as fluorescence microscopy or spectroscopy, offers the potential for multimodal imaging. This approach enables the simultaneous capture of multiple types of information from the same sample, providing a holistic view of the specimen under study. By seamlessly integrating complementary techniques, researchers can gain deeper insights into the composition and behavior of complex samples.
\end{itemize}

Meanwhile, it is worth noting that FPM offers a unique array of advantages that make it an appealing technique for a wide range of applications. Below, we provide a partial list of these potential applications.
\begin{itemize}

       \item \textbf{Biomedical applications}: It can address critical challenges in disease diagnostics, drug development, and cellular analysis by providing comprehensive data sets that facilitate a more nuanced understanding of biological specimens. The growing interest within radiology, pathology, and various medical domains centers on automating image-based diagnostic processes through machine learning techniques. FPM presents numerous advantages compared to conventional imaging methods, including heightened throughput resulting from an enhanced synthetic aperture, coupled with its phase sensitivity, which has the potential to elevate the accuracy of automated diagnostic decisions.
       
      \item \textbf{Extension in EUV regime}: While FPM has made significant strides in visible and near-infrared wavelengths, its extension into the extreme ultraviolet regimes offers unprecedented opportunities. The marriage of FPM with EUV sources holds great potential for applications in semiconductor lithography manufacturing by providing nanoscale imaging capabilities critical for quality control and defect detection. FPM can be seamlessly integrated with other EUV spectroscopy techniques, enabling simultaneous imaging and spectral analysis. This approach is invaluable for studying nanoscale materials' electronic properties, characterization, and chemical composition.

        \item \textbf{Miniaturization and accessibility}: To unlock the full potential of FPM, it must become more accessible. Future directions include making FPM systems compact, affordable, and readily available for diverse applications and settings. Designing compact and portable FPM systems that can be deployed in resource-limited settings is a critical step toward democratizing this technology. Such systems have the potential to empower researchers, doctors, and technicians in remote locations.

        \item \textbf{Automated analysis}: Machine learning techniques are increasingly finding their way into microscopy. Leveraging machine learning algorithms for automated image analysis and diagnosis represents a transformative direction for FPM. Particularly in fields like pathology and quality control, FPM combined with AI has the potential to streamline processes and improve accuracy.

\end{itemize}

\section{Conclusion}
This article outlines the development history of FPM over the past ten years and demonstrates the characteristics and limitations of FPM in detail by comparing it with other imaging technologies. And through some current application prospects, some future development directions are proposed.


All in all, FPM holds immense promise for the future of imaging technology. It effectively translates the formidable hardware demands of high-resolution microscopy imaging into challenges that can be conquered through computational means. Advancements in resolution, large FOV, phase recovery, integration with other imaging modalities, and the incorporation of machine learning techniques are poised to revolutionize various domains. By addressing challenges and fostering accessibility, FPM is on the cusp of becoming an indispensable tool in scientific research and industrial applications. The horizon for FPM gleams with potential, and its journey of transformation has only just begun.\\ \vspace{0.1em}

%%%%%%%%%%%%%%%%%%%%%%%%%%%%%%%%%%%%%%%%%%

\noindent\textbf{Author contributions:} F.X., Z.W., C.T., Y.L., and K.C. collected and reviewed references and drafted the manuscript. F.X., Z.W., C.T., and A.P. edited the manuscript. A.P. supervised and guided the whole process. A.P. funded this project. All authors have read and agreed to the published version of the manuscript.\\ \vspace{-0.5em}

\noindent\textbf{Funding:} National Natural Science Foundation of China(12104500); Key Research and Development Projects of Shaan xi Province(2023-YBSF-263). \\ \vspace{-0.5em}

\noindent\textbf{Institutional Review Board Statement}: Not applicable.\\ \vspace{-0.5em}

\noindent\textbf{Informed Consent Statement:} Not applicable. \\ \vspace{-0.5em}

\noindent\textbf{Data Availability Statement:} Not applicable. \\ \vspace{-0.5em}

\noindent\textbf{Conflicts of Interest:} The authors declare no conflict of interest.  \\ \vspace{-0.5em}

\noindent\textbf{Acknowledgments:} The authors thanks Aiye Wang for his help in the writing process. \\ \vspace{-0.1em}

\noindent\textbf{Abbreviation}

FPM, Fourier ptychographic microscopy; FOV, field-of-view; PIE, ptychographic iterative engine; 3D, three-dimensional; SIM, structured illumination microscopy; DPC, differential phase contrast; CTF, coherent transfer function; ADMM, alternating direction method of multipliers; TIE, transport of intensity equation;KK, Kramers-Kronig; SBP, spatial bandwidth product; NA, numerical aperture; DOF, depth of field; CP, coded ptychography; FP, Fourier pthography; RMS, root-mean-square; ESA, efficient synthetic aperture; RI, refractive index; DT, diffraction tomography; FPDT, Fourier ptychographic diffraction tomography; TIDT, transport of intensity diffraction tomography; NSA, non-interferometric synthetic aperture; Hela, Henrietta Lacks; CFPM, color transfer FPM; CFFPM, color-transfer filtering FPM; CTCs, circulating tumor cells; HTS, high-throughput screening; HCS, high-content screening; HCI, high-screening imaging; HT-HCI, high-throughout HCI; 

%% Only for journal Encyclopedia
%\entrylink{The Link to this entry published on the encyclopedia platform.}

\printbibliography
\end{document}

